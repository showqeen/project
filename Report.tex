\documentclass{article}
\usepackage{graphicx}
\begin{document}
\title{Report}
\date{\today}
\maketitle
\begin{itemize}


\item{\textbf{OVERVIEW:}}\\

Speech recognition, also known as automatic speech recognition (ASR),\vspace{3} computer speech recognition, or speech-to-text, is a \vspace{3}capability which enables a program to process human speech\vspace{3} into a written format.It focuses on the translation of\vspace{3} speech from a verbal format to a text one.It's working applications is\vspace{3} Google Home Assistant that can place space trivia with you and make financial transactions when requested.\vspace{3} In similar way Speech voice\vspace{3} recognition model is based on concepts of Convolution, LSTM , Attention and recognise pretrained voice with accuracy of \textbf{99.9\%}.\\

\item{\textbf{DATA:}}\\

1: Set 16KHz as sampling rate.\\
2: Record 80 utterances of each command.\\
3: Save samples of each command in different folders.\\

\item[*]{Data/forward}.
\item[*]{Data/back}.
\item[*]{Data/left}.
\item[*]{Data/right}.
\item[*]{Data/stop}.\\
\newpage

\item{\textbf{Description:}}\\

1: Using Convolutional layers ahead of LSTM is shown to improve performance in several research papers.\\

2: BatchNormalization layers are added to improve convergence rate.\\

3: Using Bidirectional LSTM is optimal when complete input is available.
But this increases the runtime two-fold.\\

4: Final output sequence of LSTM layer is used to calculate importance
of units in LSTM using a FC layer.\\

5: Then take the dot product of unit importance and output sequences of
LSTM to get Attention scores of each time step.\\

6: Take the dot product of Attention scores and the output sequences of
LSTM to get attention vector.\\

7: Add an additional FC Layer and then to output Layer with SoftMax
Activation.\\

\item{\textbf{The model is successfully built and has achieved the highest accuracy of 99.9\%}}\\

\newpage

\item{\textbf{Model Summary}}\\

\begin{figure}[h]
\includegraphics[scale=0.5]{model.png}
\end{figure}

\\

\item{\textbf{RUN:}}\\

The Code is written using Google Colab:\\

\item[1.]{Open ColabNotebook.ipynb and change Runtime to GPU.}

\item[2.]{Upload Speech-Recognition/Speech to Colab.}


\item[3.]{Change data-dir in all cells to point to Speech-Recognition/speech.}

\item[4.]{Run the cells in the same order in Notebook Test.}
\\

\item{\textbf{TEST:}}\\


1: Locate the folder where you save your model.h5 file.\\

2: Start speaking when you see mike in the bottom right pane of the task bar or see red blinking dot in the title bar.\\

\\


\item{\textbf{Language Used:}}\\

PYTHON\\

\item{\textbf{Libraries and Packages Used:}}\\

KAPRE, SCIKIT LEARN, SOUND FILE, TENSORFLOW.\\

\end{itemize}
\end{document}